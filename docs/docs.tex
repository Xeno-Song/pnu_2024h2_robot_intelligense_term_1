\documentclass{article}
\usepackage{graphicx} % Required for inserting images
\usepackage{amsmath}
\usepackage{setspace}
\usepackage{parskip}
\usepackage{amsfonts}
\usepackage{geometry}
\usepackage{physics}
\usepackage{listings}
\usepackage{kotex}

\hbadness=20000
\geometry{
 a4paper,
 left=15mm,
 right=15mm,
 top=30mm,
 bottom=40mm
 }

 \makeatletter
 \renewcommand*\env@matrix[1][\arraystretch]{%
   \edef\arraystretch{#1}%
   \hskip -\arraycolsep
   \let\@ifnextchar\new@ifnextchar
   \array{*\c@MaxMatrixCols c}}
 \makeatother

\begin{spacing}{1.125}
\begin{document}

\section*{1. Differential Drive Low-Level controller}

Twist 값이 주어졌을 때, 해당 값과 같이 Differential Drive 로봇이 움직이게 하기 위해서는 각 바퀴의 회전속도를 구하여야 한니다. 이를 위해서 Differential Drive Kinematics 공식을 응용하면 
\begin{align*}
lw + rw = x \\
rw - lw = L\omega_{z}
\end{align*}
라는 것을 쉽게 알 수 있습니다. 그리고 각각의 $lw, rw$ 는 아래와 같이 구해지고, $$lw = r \dot{\omega_{l}}, \quad rw = r \dot{\omega_{r}}$$
로봇의 제원에서 바퀴의 지름 $2r = 0.066 (m)$, 바퀴 사이의 간격 $L = 0.16(m)$ 임을 알 수 있습니다. 따라서 각 바퀴의 회전 속도는 아래의 수식을 풀어 구할 수 있습니다.
\begin{align*}
\dot{w_{l}} = \frac{x}{r} - \frac{L\omega_{z}}{2r} \\
\dot{w_{l}} = \frac{x}{r} + \frac{L\omega_{z}}{2r} 
\end{align*}
이 수식을 코드에 적용하면 아래와 같이 실행 코드를 만들 수 있습니다.
이를 실행하면 아래 링크의 영상과 같이 $"rqt_robot_steering"$ 패키지를 통해 생성된 $/cmd_vel$ 토픽에 따라 각 바퀴의 회전 속도가 계산되어 동작에 반영됩니다.

\begin{verbatim}
#include <ros/ros.h>
#include <geometry_msgs/Twist.h>
#include <trajectory_msgs/JointTrajectory.h>

geometry_msgs::Twist twist_;

void twist_subscriber_callback(const geometry_msgs::Twist::ConstPtr& twist)
{
	twist_ = *twist;
	ROS_INFO_STREAM_THROTTLE(10, "cmd_vel changed. " << twist->linear.x << ", " << twist->angular.z);
}
  
int main(int argc, char** argv)
{
	ros::init(argc, argv, "differential_controller");
	ros::NodeHandle nh;
	
	ros::Publisher joint_cmd_pub_ = nh.advertise<trajectory_msgs::JointTrajectory>("/joint_cmd", 10);
	ros::Subscriber twist_sub_ = nh.subscribe<geometry_msgs::Twist>("/cmd_vel", 10, twist_subscriber_callback);
	
	trajectory_msgs::JointTrajectory state;
	state.header.frame_id = "base";
	state.joint_names.push_back("wheel1");
	state.joint_names.push_back("wheel2");
	state.points.push_back(trajectory_msgs::JointTrajectoryPoint());
	state.points.front().positions.push_back(0.0);
	state.points.front().positions.push_back(0.0);
	state.points.front().velocities.push_back(0.0);
	state.points.front().velocities.push_back(0.0);
	
	const double wheel_distance = 0.16;
	const double wheel_radius = 0.033;
	const double wheel_circumference = wheel_radius * 2 * M_PI;
	  
	ros::Time last_update = ros::Time::now();
	
	while (ros::ok())
	{
		double linear_speed = twist_.linear.x;
		double angular_speed = twist_.angular.z;
		
		double linear_wheel_speed = linear_speed / wheel_radius;
		double angular_wheel_speed = (wheel_distance * angular_speed) / (wheel_radius * 2);
		  
		state.points.front().velocities[0] = linear_wheel_speed - angular_wheel_speed;
		state.points.front().velocities[1] = linear_wheel_speed + angular_wheel_speed;
		  
		ROS_INFO_STREAM_THROTTLE(1,
		"Velocity published. " << state.points.front().velocities[0] << ", " << state.points.front().velocities[1]);
		
		joint_cmd_pub_.publish(state);
		ros::spinOnce();
	}
}
\end{verbatim}

\section*{2. Macanum Wheels Low-Level Controller}
마찬가지로 Macanum Wheel 의 동작을 구현할 때에도 각 바퀴의 회전 속도를 구해야 합니다. 다만 Macanum Wheel 은 Kinematics 에 사용되는 행렬이 정방행렬이 아니기 때문에 Matrix Inverse 를 이용해서 Twist 에 따라 각 바퀴의 속도를 연산하는것이 어렵습니다. 대신 이 바퀴들이 각각 정방향으로 회전하였을 때 어느방향으로 움직이는 힘이 작용하는지를 행렬형태로 정리할 수 있습니다. 여기에서 1번과 3번 바퀴의 회전축이 $(0, -1, 0)$ 이라서 정방향 회전시 2번 4번 바퀴와 반대로 돌기 때문에 이를 반영하면 아래와 같은 행렬이 구해집니다.
\begin{align*}
\alpha &= l_{1} + l_{2} = 0.1125 + 0.0925 = 0.205 \\
\begin{bmatrix}
\omega_{1} \\ \omega_{2} \\ \omega_{3} \\ \omega_{4}
\end{bmatrix} &= \frac{1}{r}\begin{bmatrix}
-1 & -1 & -\alpha \\
1 & -1 & -\alpha \\
-1 & 1 & -\alpha \\
1 & 1 & -\alpha
\end{bmatrix}\begin{bmatrix}
v_{x} \\ v_{y} \\ \omega_{z}
\end{bmatrix}
\end{align*}

따라서 우측 벡터에 원하는 속도와 각속도를 넣으면 각 휠의 동작 속도를 구할 수 있게 됩니다.
이를 코드로 구현하면 아래와 같이 작성됩니다.

\begin{verbatim}
#include <ros/ros.h>
#include <Eigen/Eigen>
#include <geometry_msgs/Twist.h>
#include <trajectory_msgs/JointTrajectory.h>
  
geometry_msgs::Twist twist_;
  
void twist_subscriber_callback(const geometry_msgs::Twist::ConstPtr& twist)
{
	twist_ = *twist;
	ROS_INFO_STREAM_THROTTLE(10, "cmd_vel changed. " << twist->linear.x << ", " << twist->angular.z);
}
  
int main(int argc, char** argv)
{
	ros::init(argc, argv, "mecanum_drive");
	ros::NodeHandle nh;
	  
	ros::Publisher joint_cmd_pub_ = nh.advertise<trajectory_msgs::JointTrajectory>("/joint_cmd", 10);
	ros::Subscriber twist_sub_ = nh.subscribe<geometry_msgs::Twist>("/cmd_vel", 10, twist_subscriber_callback);
	  
	trajectory_msgs::JointTrajectory state;
	state.header.frame_id = "base";
	state.joint_names.push_back("wheel1");
	state.joint_names.push_back("wheel2");
	state.joint_names.push_back("wheel3");
	state.joint_names.push_back("wheel4");
	state.points.push_back(trajectory_msgs::JointTrajectoryPoint());
	state.points.front().velocities.push_back(0.0);
	state.points.front().velocities.push_back(0.0);
	state.points.front().velocities.push_back(0.0);
	state.points.front().velocities.push_back(0.0);
	  
	// mecanum wheel radius : 0.5 (subwheel was not considered, but it may ok)
	// l1 = 0.1125, l2 = 0.0925
	const double alpha = 0.1125 + 0.0925;
	Eigen::Matrix<double, 4, 3> macanum_kinematics_matrix;
	macanum_kinematics_matrix <<
	-1, -1, -alpha,
	1, -1, -alpha,
	-1, 1, -alpha,
	1, 1, -alpha;
	macanum_kinematics_matrix /= 0.05; // wheel radius
  
	while (ros::ok())
	{
		Eigen::Vector3d velocities(twist_.linear.x, twist_.linear.y, twist_.angular.z);
		Eigen::Vector4d angular_velocities = macanum_kinematics_matrix * velocities;
		state.points.front().velocities[0] = angular_velocities(0);
		state.points.front().velocities[1] = angular_velocities(1);
		state.points.front().velocities[2] = angular_velocities(2);
		state.points.front().velocities[3] = angular_velocities(3);
		  
		ROS_INFO_STREAM_THROTTLE(1,
		"Velocity published. " <<
		state.points.front().velocities[0] << ", " << state.points.front().velocities[1] << ", " <<
		state.points.front().velocities[2] << ", " << state.points.front().velocities[3]);
		  
		joint_cmd_pub_.publish(state);
		  
		ros::spinOnce();
	}
}
\end{verbatim}

해당 코드를 빌드하여 실행하면 아래와 같이 실행 결과를 얻을 수 있습니다.

\section*{3. 4R Robot Arm IK}
우선 로봇 팔의 역기구학을 풀기 위해서 파라미터를 아래와 같이 정리하였습니다. 
\begin{align*}
d_{1} = \begin{bmatrix}
0 \\ 0 \\ 0.0595
\end{bmatrix}, \quad l_{2} = \begin{bmatrix}
0.024 \\ 0 \\ 0.128
\end{bmatrix}\quad l_{3} = \begin{bmatrix}
0.124 \\ 0 \\ 0
\end{bmatrix}, \quad l_{4} = \begin{bmatrix}
0.0817 \\ 0 \\ 0
\end{bmatrix}
\end{align*}
여기에 목표위치를 $P_{t}$ 라고 두었을 때, $P_{t}$ 위치와 회전으로 로봇 팔이 정렬되기 위해서는 $P_{t}$ 의 $z$ 축 방향에 로봇의 마지막 Joint 가 정렬되어야 합니다. 따라서 마지막 joint 이후의 link 인 $l_{4}$ 의 길이 만큼 $P_{t}$ 위치를 기준으로 한 $-z$ 방향으로 나머지 $l_{1}, l_{2}, l_{3}$ 를 배치할 수 있다면 목표 지점의 IK 값을 구할 수 있습니다.
따라서 우선 Offset 된 위치를 아래와 같이 계산하였습니다. 
\begin{align*}
    P_{t} &= \begin{bmatrix}
        R_{t}& p_{t} \\
        0 & 1
    \end{bmatrix} = \begin{bmatrix}
        & & & \\
        \vec{x_{t}} & \vec{y_{t}} & \vec{z_{t}} & p_{t} \\
        & & &  \\
        0 & 0 & 0 & 1
    \end{bmatrix} \\
    p_{w} &= p_{t} + l_{4}\vec{z_{t}}
\end{align*}
남은 $l_{1}, l_{2}, l_{3}$ 중 $z$ 축을 회전축으로 가지는 회전축은 1축밖에 없고, 따라서 우선 $x, y$ 성분을 이용해 로봇 팔의 방향을 정렬할 수 있습니다. $$q_{1} = \text{atan2}(p_{wy}, p_{wx})$$
이후 남은 두 Y 축 회전 Joint 를 이용해 $p_{w}$ 까지의 자세를 만들어야 하고, 이는 $l_{2}, l_{3}$ 의 길이와 $p_{w}$ 까지의 거리를 가지고 코사인법칙을 이용하면 각각의 각을 구할 수 있습니다.
\begin{align*}
m &= ||((||p_{wx}, p_{wy})||^{2}, p_{wz} - d_{1})||_{2} \\ \\
\phi &= \text{atan2}(l_{2x}, l_{2z})\\
\gamma &= \frac{l_{2}^2 + l_{3}^2 - m^2}{2l_{2}l_{3}} \\
\beta &= \frac{m^2 + l_{2}^2 - l_{3}^2}{2ml_{2}} \\
q_{2} &= \frac{\pi}{2} - \alpha - \beta - \phi \\
q_{3} &= \phi - \gamma - \frac{\pi}{2}
\end{align*}
여기까지 완료되면 $q_{4}$ 값만 구하면 되는데, 우선 이를 위해서 $q_{1}, q_{2}, q_{3}$ 가 반영된 FK 를 먼저 해석해야 합니다. $$P_{q_{1}q_{3}} = \text{trans}(d_{1})\times\text{rotz}(q_{1})\times\text{roty}(q_{2}+\phi)\times\text{trans}(l_{2})\times\text{rotz}(q_{3}-\phi) \times\text{trans}(l_{3})$$
FK 결과가 가지고 있는 Rotation Matrix 와 목표 위치의 Rotation Matrix 를 이용해 $q_{4}$ 의 각도를 구합니다. 
\begin{align*}
P_{q_{1}q_{3}}&=\begin{bmatrix}
R_{q_{1}q_{3}} & p_{q_{1}q_{3}} \\
0 & 1
\end{bmatrix} \\
R_{wt} &= R_{q_{1}q_{3}}^T R_{t} \\
q_{4} &= \text{atan2}(R_{wt}^{(0, 2)}, R_{wt}^{(0, 0)})
\end{align*}
따라서 각 joint 의 값인 $q_{1}, q_{2}, q_{3}, q_{4}$ 를 구할 수 있게 되고, 이를 코드로 반영하면 아래와 같은 코드로 만들 수 있습니다. 다만 실제 구현 코드의 대부분이 목표위치를 지정하기 위한 Interactive Marker 를 표시하는 코드이기 때문에 해당 부분의 코드를 제외하고 실제 IK 부분의 코드만 리포트에 첨부하고, 전체 코드는 아래의 github repository 에 업로드 해 두었습니다.

\begin{verbatim}
...

geometry_msgs::Pose z_rotation_pose_;
geometry_msgs::Pose target_pose_;
  
std::vector<double> arm_target;

inline double get_center_angle_using_three_length(double a, double b, double c)
{
	return std::acos((a * a + b * b - c * c) / (2 * a * b));
}
  
Eigen::Matrix4d pose_to_marix(geometry_msgs::Pose pose)
{
	Eigen::Matrix4d matrix = Eigen::Matrix4d::Identity();
	matrix.block<3, 3>(0, 0) = Eigen::Quaterniond(pose.orientation.w, pose.orientation.x, pose.orientation.y, pose.orientation.z).toRotationMatrix();
	matrix(0, 3) = pose.position.x;
	matrix(1, 3) = pose.position.y;
	matrix(2, 3) = pose.position.z;
	return matrix;
}
  
void solve_robot_arm_ik()
{
	Eigen::Vector3d link3_to_4(0.024, 0, 0.128);
	  
	const double d1 = 0.0595;
	const double link3_length = std::sqrt(0.024 * 0.024 + 0.128 * 0.128);
	const double link4_length = 0.124;
	const double link5_length = 0.0817;
	  
	const double link3_angle = std::atan2(0.024, 0.128);
	  
	Eigen::Matrix4d target_transform = Eigen::Matrix4d::Identity();
	target_transform.block<3, 3>(0, 0) = pose_to_marix(z_rotation_pose_).block<3, 3>(0, 0);
	target_transform = target_transform * pose_to_marix(target_pose_);

	Eigen::Vector3d wrist_position = target_transform.block<3, 1>(0, 3) - target_transform.block<3, 1>(0, 2) * link5_length;
	  
	double x_prime = Eigen::Vector2d(wrist_position.block<2, 1>(0, 0)).norm();
	Eigen::Vector2d m_vec = Eigen::Vector2d(x_prime, wrist_position.z() - d1);
	double m = m_vec.norm();
	double alpha = std::atan2(m_vec.y(), m_vec.x());
	  
	double gamma = get_center_angle_using_three_length(link3_length, link4_length, m);
	double beta = get_center_angle_using_three_length(m, link3_length, link4_length);
	 
	if (gamma > 0.0) gamma -= M_PI; // to make value in range of -M_PI ~ M_PI
	if (beta < 0.0) beta += M_PI;   // to make value in range of -M_PI ~ M_PI
	  
	double q1 = std::atan2(wrist_position.y(), wrist_position.x());
	double q2 = M_PI / 2 - alpha - beta - link3_angle;
	double q3 = -(gamma - link3_angle + M_PI / 2);
	Eigen::Matrix4d link4_position = Eigen::Affine3d(
		Eigen::Translation3d(0, 0, d1) *
		Eigen::AngleAxisd(q1, Eigen::Vector3d::UnitZ()) *
		Eigen::AngleAxisd(q2 + link3_angle, Eigen::Vector3d::UnitY()) *
		Eigen::Translation3d(0, 0, link3_length) *
		Eigen::AngleAxisd(q3 - link3_angle, Eigen::Vector3d::UnitY()) *
		Eigen::Translation3d(0, 0, link4_length)).matrix();
	  
	Eigen::Matrix3d orientation = link4_position.block<3, 3>(0, 0).transpose() * target_transform.block<3, 3>(0, 0);
	double q4 = std::atan2(orientation(0, 2), orientation(0, 0)) - M_PI / 2;
	  
	std::cout << "Solution : " << q1 << ", " << q2 << ", " << q3 << ", " << q4 << std::endl;
	std::vector<double> solution;
	solution.push_back(q1);
	solution.push_back(q2);
	solution.push_back(q3);
	solution.push_back(q4);
	arm_target = solution;
}

...

void joint_state_callback(const sensor_msgs::JointState::ConstPtr& state)
{
	if (arm_target.size() == 0) return;
	  
	trajectory_msgs::JointTrajectory joint_trajectory;
	joint_trajectory.header.frame_id = "base";
	joint_trajectory.joint_names.push_back("Arm1");
	joint_trajectory.joint_names.push_back("Arm2");
	joint_trajectory.joint_names.push_back("Arm3");
	joint_trajectory.joint_names.push_back("Arm4");
	joint_trajectory.points.push_back(trajectory_msgs::JointTrajectoryPoint());
	joint_trajectory.points.front().positions.push_back(arm_target.at(0));
	joint_trajectory.points.front().positions.push_back(arm_target.at(1));
	joint_trajectory.points.front().positions.push_back(arm_target.at(2));
	joint_trajectory.points.front().positions.push_back(arm_target.at(3));
	  
	joint_cmd_pub_.publish(joint_trajectory);
}

int main(int argc, char** argv)
{
	ros::init(argc, argv, "robot_arm_drive");
	ros::NodeHandle nh;

	...

	joint_cmd_pub_ = nh.advertise<trajectory_msgs::JointTrajectory>("/joint_cmd", 10);
	ros::Subscriber joint_state_sub_ = nh.subscribe<sensor_msgs::JointState>("/joint_states", 10, joint_state_callback);
	  
	ros::spin();
	
	...
	return 0;
}
\end{verbatim}

해당 코드를 빌드하여 실행하면 아래 영상과 같은 결과를 얻을 수 있습니다.


\end{document}
\end{spacing}
